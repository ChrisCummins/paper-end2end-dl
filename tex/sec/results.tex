\section{Experimental Results} \label{sec:results}

We evaluate the effectiveness of DeepTune for two distinct optimization tasks: predicting the optimal device to run a given program, and predicting thread coarsening factors.

We first compare DeepTune against two expert-tuned predictive models, showing that DeepTune outperforms the state-of-the-art in both cases. We then show that by leveraging knowledge learned from training DeepTune for one heuristic, we can boost training for the other heuristic, further improving performance. Finally, we analyze the working mechanism of DeepTune.


\subsection{Case Study A: OpenCL Heterogeneous Mapping}

Selecting the optimal execution device for OpenCL kernels is essential for maximizing performance. For a CPU/GPU heterogeneous system, this presents a binary choice. In this experiment, we compare our approach against a static single-device approach and the Grewe \emph{et al.\ }predictive model. The \emph{static mapping} selects the device which gave the best average case performance over all the programs. On the AMD platform, the best-performing device is the CPU; on the NVIDIA platform, it is the GPU.

Figure~\ref{fig:cgo-accuracy} shows the accuracy of both predictive models and the static mapping approach for each of the benchmark suites. The static approach is accurate for only 58.8\% of cases on AMD and 56.9\% on NVIDIA. This suggests the need for choosing the execution device on a per program basis. The Grewe \emph{et al.\ }model achieves an average accuracy of 73\%, a significant improvement over the static mapping. By automatically extracting useful feature representations from the source code, DeepTune gives an average accuracy of 82\%, an improvement over both schemes.

Using the static mapping as a baseline, we compute the relative performance of each program using the device selected by the Grewe \emph{et al.\ }and DeepTune models. Figure~\ref{fig:cgo-speedup} shows these speedups. Both predictive models significantly outperform the static mapping; the Grewe \emph{et al.\ }model achieves an average speedup of $2.91\times$ on AMD and $1.26\times$ on NVIDIA (geomean $1.18\times$). In 90\% of cases, DeepTune matches or outperforms the predictions of the Grewe \emph{et al.\ }model, achieving an average speedup of $3.34\times$ on AMD and $1.41\times$ on NVIDIA (geomean $1.31\times$). This 14\% improvement in performance comes at a greatly reduced cost, requiring no intervention by humans.

\begin{figure}
  \centering %
  \includegraphics[width=\columnwidth]{img/cgo-acc}%
  \caption{%
  Accuracy of optimization heuristics for heterogeneous device mapping,
  aggregated by benchmark suite. The optimal static mapping achieves 58\%
  accuracy. The Grewe \emph{et al.\ }and DeepTune predictive models achieve
  accuracies of 73\% and 84\%, respectively.%
  }
  \label{fig:cgo-accuracy}
\end{figure}

\begin{figure*}
  \centering %
  \includegraphics[width=\textwidth]{img/cgo-speedup}%
  \vspace{-1em}
  \caption{%
  Speedup of predicted heterogeneous mappings over the best static mapping for
  both platforms. In (a) DeepTune achieves an average speedup of 3.43x over
  static mapping and 18\% over Grewe \emph{et al}. In (b) the speedup is 1.42x
  and 13\% respectively.%
  }
  \label{fig:cgo-speedup}
  %
\end{figure*}



\subsection{Case Study B: OpenCL Thread Coarsening Factor}

\begin{figure*}
  \centering %
  \includegraphics[width=\textwidth]{img/pact-speedup}%
  \vspace{-1em}
  \caption{%
  Speedups of predicted coarsening factors for each platform. DeepTune
  outperforms Magni \emph{et al} on three of the four platforms. Transfer
  learning improves DeepTune speedups further, by 16\% on average.%
  }%
  \label{fig:pact-speedup}
  %
\end{figure*}


Exploiting thread coarsening for OpenCL kernels is a difficult task. On average, coarsening slows programs down. The speedup attainable by a perfect heuristic is only $1.36\times$.

Figure~\ref{fig:pact-speedup} shows speedups achieved by the Magni \emph{et al.\ }and DeepTune models for all programs and platforms. We use as baseline the performance of programs without coarsening. On the four experimental platforms (AMD HD 5900, Tahiti 7970, NVIDIA GTX 480, and Tesla K20c), the Magni \emph{et al.\ }model achieves average speedups of $1.21\times$, $1.01\times$, $0.86\times$, and $0.94\times$, respectively. DeepTune outperforms this, achieving speedups of $1.10\times$, $1.05\times$, $1.10\times$, and $0.99\times$.

Some programs --- especially those with large divergent regions or indirect memory accesses --- respond very poorly to coarsening. No performance improvement is possible on the \texttt{mvCoal} and \texttt{spmv} programs. Both models fail to achieve positive average speedups on the NVIDIA Tesla K20c, because thread coarsening does not give performance gains for the majority of the programs on this platform.

The disappointing results for both predictive models can be attributed to the small training program set used by Magni \emph{et al.\ }(only 17 programs in total). As a result, the models suffer from sparse training data. Prior research has shown that data sparsity can be overcome using additional programs; in the following subsection we describe and test a novel strategy for training optimization heuristics on a small number of programs by exploiting knowledge learned from other optimization domains.


\subsection{Transfer Learning Across Problem Domains}\label{subsec:tl}

There are inherent differences between the tasks of building heuristics for heterogeneous mapping and thread coarsening, evidenced by the contrasting choices of features and models in Grewe \emph{et al.\ }and Magni \emph{et al.\ } However, in both cases, the first role of DeepTune is to extract meaningful abstractions and representations of OpenCL code. Prior research in deep learning has shown that models trained on similar inputs for different tasks often share useful commonalities. The idea is that in neural network classification, information learned at the early layers of neural networks (i.e. closer to the input layer) will be useful for multiple tasks. The later the network layers are (i.e. closer to the output layer), the more specialized the layers become~\cite{Zeiler2014}.

We hypothesized that this would be the case for DeepTune, enabling the novel transfer of information \emph{across different optimization domains}. To test this, we extracted the language model --- the \texttt{Embedding}, and \texttt{LSTM\_\{1,2\}} layers --- trained for the heterogeneous mapping task and \emph{transferred} it over to the new task of thread coarsening. Since DeepTune keeps the same design for both optimization problems, this is as simple as copying the learned weights of the three layers. Then we trained the model as normal.

As shown in Figure~\ref{fig:pact-speedup}, our newly trained model, DeepTune-TL has improved performance for 3 of the 4 platforms: $1.17\times$, $1.23\times$, $1.14\times$, $0.93\times$, providing an average 12\% performance improvement over Magni \emph{et al.}  In 81\% of cases, the use of transfer learning matched or improved the optimization decisions of DeepTune, providing up to a 16\% improvement in per platform performance.

On the NVIDIA Tesla K20c, the platform for which no predictive model achieves positive average speedups, we match or improve performance in the majority of cases, but over-coarsening on three of the programs causes a modest reduction in average performance. We suspect that for this platform, further performance results are necessary due to its unusual optimization profile.


\subsection{DeepTune Internal Activation States}

\begin{figure*}
  \centering %
  \includegraphics[width=\textwidth]{img/viz}%
  \vspace{-1em}
  \caption{%
  Visualizing the internal state of DeepTune when predicting coarsening factor
  for Parboil's \texttt{mriQ} benchmark on four different architectures. The
  activations in each layer of the four models increasingly diverge the lower
  down the network.%
  }
  \label{fig:viz}
  %
\end{figure*}


We have shown that DeepTune automatically outperforms state-of-the-art predictive models for which experts have invested a great amount of time in engineering features. In this subsection we attempt to illuminate the inner workings, using a single example from Case Study B: predicting the thread coarsening factor for Parboil's \texttt{mriQ} benchmark on four different platforms.

Figure~\ref{fig:viz} shows the DeepTune configuration, with visual overlays showing the internal state. From top to bottom, we begin first with the input, which is the 267 lines of OpenCL code for the \texttt{mriQ} kernel. This source code is preprocessed, formatted, and rewritten using variable and function renaming, shown in Figure~\ref{fig:viz}b. The rewritten source code is tokenized and encoded in a $1$-of-$k$ vocabulary. Figure~\ref{fig:viz}c shows the first 80 elements of this encoded sequence as a heatmap in which each cell's color reflects its encoded value. The input, rewriting, and encoding is the same for each of the four platforms.

The encoded sequences are then passed into the Embedding layer. This maps each token of the vocabulary to a point in a 64 dimension vector space. Embeddings are learned during training so as to cluster semantically related tokens together. As such, they may differ between the four platforms. Figure~\ref{fig:viz}d shows a PCA projection of the embedding space for one of the platforms, showing multiple clusters of tokens. By honing in on one of the clusters and annotating each point with its corresponding token, we see that the cluster contains the semantically related OpenCL address space modifiers \texttt{\_\_private}, \texttt{\_\_global}, and \texttt{\_\_read\_only}.

Two layers of 64 LSTM neurons model the sequence of embeddings, with the neuron activations of the second layer being used to characterize the entire sequence. Figure~\ref{fig:viz}e shows the neurons in this layer for each of the four platforms, using a red-blue heatmap to visualize the intensity of each activation. Comparing the activations between the four platforms, we note a number of neurons in the layer with different responses across platforms. This indicates that the language model is partly specialized to the target platform.

As information flows through the network, the layers become progressively more specialized to the specific platform. We see this in Figure~\ref{fig:viz}f, which shows the two layers of the heuristic model. The activations within these increasingly diverge. The mean variance of activations across platforms increases threefold compared to the language model, from 0.039 to 0.107. Even the activations of the AMD HD 5900 and AMD Tahiti 7970 platforms are dissimilar, despite the final predicted coarsening factor for both platforms being the same. In Figure~\ref{fig:viz}g we take the largest activation of the output layer as the final predicted coarsening factor. For this particular program, a state-of-the-art model achieves 54\% of the maximum performance. DeepTune achieves 99\%.
